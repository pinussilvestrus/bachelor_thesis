\myabstract{
	
	Einer der wichtigsten Vorhaben der deutschen Schul-Cloud ist es, verschiedenartige Unterrichtsmaterialien für jedermann zugänglich und auf mehreren Geräten verfügbar zu machen. Es soll möglich sein, Dateien kollaborativ zu teilen und im Unterricht ohne zusätzliche Software einzusetzen.
	
	Das Ziel der vorliegenden Bachelorarbeit war es, eine verteilte und damit flexible Dateiverwaltung für das Schul-Cloud zu entwerfen. Es wurden bereits bewährte File Storage Anbieter evaluiert, Studien im Bereich digitalen Lernens ausgewertet und daraus ein dynamisches Konzept für eine Dateiablage entwickelt. Verschiedene Implementierungen im derzeitigen Schul-Cloud Server und Client wurden aufgezeigt und mit anderen Lösungen verglichen. Zudem wurde eine Reihe von Anschlussmöglichkeiten für die Dateiverwaltung aufgezählt.
	
}{
	One of the major purposes of German Schul-Cloud is to make various teaching materials accessible to everyone and to make them available on several devices. It should be possible to share files collaboratively and use them in lessons without additional software.
	
	The goal of the present Bachelor thesis was to design a distributed flexible storage system for the Schul-Cloud. Existing file storage providers have been evaluated, studies in the area of digital learning have been checked and a dynamic concept for a file storage has been developed. Several implementations in the current Schul-Cloud server and client have been presented and compared with other solutions. In addition, a number of connectivity options for file management were enumerated.
}