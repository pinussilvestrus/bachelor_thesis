\section{Zusammenfassung}
\label{sec:conclusion}

Eines der übergeordneten Ziele der deutschen Schul-Cloud ist es, Lernmaterial für alle und überall zugänglich zu machen. Diese Arbeit wollte erreichen, eine flexible Dateiverwaltung für dieses Vorhaben zu entwerfen und zu implementieren. Die Anforderungen wurden zu Beginn des Projekts klar definiert, so dass das Konzept auf diesen aufbauen konnte.

 Es wurde sich für ein in Schulen aufgeteiltes Bucket-System entschieden, um Dateien logisch zu trennen. Diese werden in die einzelnen Kontexte des schulischen Alltags untergliedert, d.h. Kurse bzw. Fächer sowie Klassen und einzelne Individuen. Das Strategy-Pattern schaffte zudem die Möglichkeit, Buckets auf mehrere Server zu verteilen. Das Ziel einer grundlegend verteilten Dateiverwaltung wurde somit erfüllt. Viele Konzepte für die Dateiverwaltung wurden vom AWS S3-Standard abgeleitet. Da viele anderen Systeme wie zum Beispiel Microsoft Azure \footnote{Microsoft Azure - \url{https://azure.microsoft.com/de-de/}} auf den Standard aufbauen, liegt es nahe, diesen im Schul-Cloud System zu benutzen. Das Strategy-Pattern ermöglicht es trotzdem, von diesem abzuweichen und ganz eigene Strategien zu implementieren.

Der Open-Source Status der Schul-Cloud soll hierbei helfen, zusammen mit Schülern und Lehrern die Schul-Cloud weiter zu verbessern. Strategien können nach und nach dazu implementiert werden und ins bestehende System eingepflegt werden. Eine Schule hat so zum Beispiel die Möglichkeit, für ihren Schul-Server eine eigene Anbindung zu schreiben, ohne sich zu stark an den Schul-Cloud Standard zu richten. Abschließend lässt sich sagen, dass die entworfenen Konzepte in der derzeit laufenden Pilotphase getestet und Rückmeldungen eingeholt werden müssen.

\clearpage
