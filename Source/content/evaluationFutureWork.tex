\section{Evaluierung und zukünftige Arbeit}
\label{sec:evaluation}

Es gilt nun, die erarbeiteten Konzepte mit den Anforderungen abzugleichen. In Abschnitt \ref{sec:relatedwork} wurde klar, dass zum einen die Ablage von Dateien in mehreren Kontexten möglich sein soll. Dies wurde erreicht, indem man den Grundaufbau der Dateiverwaltung genau auf diesem Prinzip aufbauen lies. Die einzelnen Schul-Cloud Dateien sind genau einem Schul-Bucket zugeteilt und dort in den zugehörigen Kontexten, d.h. persönliche Dateien sowie Kurs - und Klassendateien, unterteilt. Außerdem sollte das erarbeitete Konzept ermöglichen, bereits bestehende Dateisysteme einzubinden, um Schulen die Arbeit beim Einpflegen der bestehenden Dateibestände zu gut es geht zu erleichtern. Dies wurde mit dem Strategy-Pattern erreicht (Abschnitt \ref{sec:strategypatternconcept}). Jede Schule kann im Administrationsbereich des Frontends selbst entscheiden, in welcher Art des Buckets die Dateien gelegt werden können. Dies kann eine AWS S3-Instanz, ein ownCloud-Server oder ein FTP-Zugriff sein. Wichtig ist nur, dass die nötige Strategie dafür implementiert ist. Dieses Konzept ist noch nicht vollständig im bestehenden System implementiert. Für die Pilotphase wird vom Projektteam eine geteilte S3 - Instanz bereitgestellt. Die URL sowie ein Import für den Schul-Bucket sollte in Zukunft noch einstellbar sein. Dies kann ebenfalls im Schul-Modell eingebaut werden. Außerdem gilt es eine Lösung für den Fall zu finden, dass eine Schule ein bestehenden Bucket benutzen möchte, der nicht dem Grundaufbau entspricht. Ein Lösung wäre hier eine Zwischenablage, so dass der Administrator die Dateien richtig verteilen kann. Eine weitere Anforderung war es, Dateien in den Unterricht und bei Hausaufgaben einbauen zu können. Diese Möglichkeit wurde im Abschnitt \ref{sec:userinterface} mittels ckEditor gezeigt. Das gewählte Konzept erfüllt also alle genannten Anforderungen.

Im Verlaufe des Bachelorprojekts wurde das Konzept der Dateiverwaltung oft überdacht. Beispielsweise war die \textit{path} - Variable, die in vielen Funktionen benutzt wird, ehemals in zwei Variablen aufgeteilt. Es gab einmal den Kontext-Pfad und den Dateinamen. Das Zusammenfassen bzw. Konkatenieren hatte den Vorteil, dass Funktionsdefinitionen übersichtlicher und die Benutzung einfacher wurde. Es wurde auch über die Umstellung auf eine flache Dateistruktur nachgedacht. Dies würde bedeuten, dass alle Dateien innerhalb eines Buckets in einem einzigen Ordner untergebracht sind. Die Aufteilung in verschiedene Kontexte würde ein Feathers Proxy-Service im Schul-Cloud-Server übernehmen, welcher alle Daten zu den Dateien verwaltet.. Dieser würde sich außerdem um Berechtigungen kümmern. Viele Funktionen könnten vom File Storage Provider abstrahiert werden. Zum Beispiel müsste man beim Verschieben einer Datei nur den Datenbankeintrag ändern. Die Vorteile sind klar ersichtlich, man hätte im Client einen geringeren Aufwand bei der Ermittlung der richtigen Kontext-Pfade. Sehr viel Arbeit würde in den Server wandern. Man könnte dafür den bereits bestehenden \textit{FilePermissionService} erweitern. Hier liegt aber auch ein Nachteil. Der Server würde ein Großteil der Arbeit erledigen. Man hätte zwar immer noch ein Strategy-Pattern, jedoch würde man viele Funktionen selbst implementieren müssen. Das Ziel der Arbeit war es zudem, auf bestehende Systeme zurück zu greifen zu können, um sich Arbeit zu sparen. Die flache Dateistruktur scheint viele Vorteile zu haben, jedoch hat man momentan ein funktionierendes Konzept für die Pilotphase und es bleibt zu evaluieren, wie dieses von den Testern aufgenommen wird. Wenn es zu Problemen kommt, sollte man über diese Alternative nachdenken.

Nach Beendigung des Bachelorprojekts waren noch längst nicht alle Aufgaben abgearbeitet. Dank der bereits laufenden Pilotphase sind viele interessante Wünsche und Anforderungen dazugekommen, welche es sich in Zukunft zu implementieren lohnt. Beispielsweise sollten die Zugriffsrechte erweitert werden. Alle für einen Nutzer geteilten Dateien sollen über einen zusätzlichen Menüpunkt zugreifbar sein. Außerdem wird ein Synchronisations-Client für den Desktop-Einsatz gewünscht. Ähnlich wie bei ownCloud soll es möglich sein, die Schul-Cloud Dateien auf der lokalen Dateiablage des genutzten Betriebssystems zu verwalten. Außerdem ist ein Service für die Vorschau von Dateien in Planung, der aber außerhalb des Bachelorprojekts entwickelt wird. Zur Verbesserung der Nutzerfreundlichkeit sollten Funktionalitäten wie das Suchen und Sortieren von Dateien implementiert werden. Ein erweiterte Admin-Ansicht auf die Dateiverwaltung wäre ebenfalls wünschenswert, um etwa Berechtigungen zu verwalten. Wichtiger Punkt ist außerdem, Dateien in den sogenannten \textit{Lern-Store} \cite{online:zehnforderungen} der Schul-Cloud einzubinden. Sehr engagierte Lehrer sollen die Möglichkeit haben, aufbereitetes Lernmaterial anderen Lehrern zugänglich machen. 

\clearpage
