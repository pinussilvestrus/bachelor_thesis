\section{Introduction}
\label{sec:intro}

Die deutsche Schul-Cloud soll "dabei helfen, die digitale Transformation in Schulen zu meistern und den faecheruuebergreifenden Unterricht mit digitalen Inhalten zu bereichern" [1]. Ein Schwerpunkt ist hierbei das Arbeiten mit verschiedenen Dateiformaten im Unterrichtskontext. Dazu soll die Schul-Cloud eine Möglichkeit schaffen, eigene Dateien zu verwalten und sie unter Lehrern und Schülern zu teilen. Dafür soll keine neue Dateiverwaltungstechnologie entwurfen werden, sondern auf bestehende Systeme zurückgeführt werden. Viel mehr soll für den Nutzer der Schul-Cloud möglich sein, ein bestehendes Dateisystem weiter zu nutzen und in die Schul-Cloud zu integrieren. Außerdem soll es möglich sein, Dateien auf verschiedenen Systemen zu lagern, um eine flexible Architektur zu schaffen.

In dieser Bachelor-Arbeit wird eine solche Architektur für ein verteiltes Dateiverwaltungssystem beschrieben und entworfen. 

\clearpage
