\section{Einleitung}
\label{sec:intro}

Die deutsche Schul-Cloud soll ``dabei helfen, die digitale Transformation in Schulen zu meistern und den fächerübergreifenden Unterricht mit digitalen Inhalten zu bereichern'' \footnote{ Technischer Bericht (Seite 5) \cite{paper:technischerbericht}}. Ein Schwerpunkt ist hierbei das Arbeiten mit verschiedenen Dateiformaten im Unterrichtskontext. Dazu soll die Schul-Cloud eine Möglichkeit schaffen, eigene Dateien zu verwalten und diese unter Lehrern und Schülern zu teilen. Die Bildung einer Dateiablage ist nach neueren Analysen fester Bestandteil von ``technisch-organisatorische[n] Kernanforderungen" \cite{paper:breiterstolpmannzeising2015} für den digitalen Unterricht. Dafür soll keine neue Dateiverwaltungstechnologie entworfen, sondern auf bestehende Systeme zurückgegriffen werden. Vielmehr soll es für den Nutzer der Schul-Cloud möglich sein, eine bestehende Dateiablage weiter zu nutzen und in die Schul-Cloud zu integrieren. Außerdem ist es erforderlich, Dateien auf verschiedenen Systemen zu lagern, um eine flexible Architektur zu schaffen. Dies ist vor allem dann nötig, wenn große Datenmengen von sehr vielen Schulen, Schülern und Lehrern entstehen. Somit ist es wichtig, eine Lösung in Form eines verteilten File Storage zu finden. 

In dieser Bachelor-Arbeit wird eine solche Architektur für ein verteiltes Dateiverwaltungssystem beschrieben und entworfen. Begonnen wird mit einer \textbf{Vorbetrachtung} (2), in welcher die bestehende Situation von Dateiverwaltung im Schulunterricht diskutiert wird. Auch werden bereits bestehende Konzepte anderer Arbeiten analysiert und Vorbilder ausgemacht. Danach wird ein \textbf{Konzept} (3) für ein verteiltes System beschrieben und anschließend eine mögliche \textbf{Implementierung} (4)  aufgezeigt. In dieser wird eine Anbindung von File Storage-Providern im Schul-Cloud Server und Client beschrieben. Diese Anbindungsmöglichkeit wird anschließend \textbf{evaluiert} und mögliche \textbf{Anschlussszenarien} aufgezeigt (5). Zum Ende werden die Ergebnisse \textbf{zusammengefasst} (6).

\clearpage
